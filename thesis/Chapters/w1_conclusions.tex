%!TEX root = ../thesis.tex
% створюємо Висновки до всієї роботи
У ході роботи був проведений аналіз наукових джерел за тематикою моделей природних мов та моделей відкритого тексту, формальної теорії мов та граматик, теорії прихованих моделей Маркова,

В роботі вперше запропоновано модель природної мови на основі прихованих моделей Маркова другого порядку зі стеком, разом з цим було доведено теорему про еквівалентність стохастичних регулярних граматик та прихованих моделей Маркова
другого порядку за спостереженнями та теорему про еквівалентність стохастичних контекстно-вільних граматик
та прихованих моделей Маркова другого порядку за спостереженнями зі стеком.
Дані результати є важливими теоретичним фундаментом для розробки нових алгоритмів навчання стохастичних
контекстно-вільних граматик.даної

Також запропоновано модель відкритого тексту на основі створеної прихованої моделі Маркова другого порядку за спостереженнями зі стеком, що дозволяє ефективне моделювання граматичної структури природної мови, оскільки модель еквівалентна стохастичним контекстно-вільним граматикам, котрі досить добре можуть описувати граматичну структуру та деякі семантичні властивості мови.

Було розроблено програмний пакет, що дозволяє моделювати стохастичні
контекстно-вільні граматики за допомогою прихованих моделей Маркова другого порядку за спостереженнями зі стеком.
Зокрема в ньому реалізовано описаний поліноміальний алгоритм зведення стохастичної контекстно-вільної граматики
до прихованої моделі Маркова другого порядку за спостереженнями зі стеком.

Емпіричним шляхом було перевірено узгодженість моделі з стохастичними контекстно-вільними граматиками на прикладі довільної граматики.

Розглянута в роботі тематика може бути продовжена в контексті розробки нових алгоритмів навчання стохастичних контекстно-вільних
граматик, оскільки наявні алгоритми мають досить велику обчислювальну складність, а нова модель є простішою у алгоритмічному плані. Також результати роботи можуть бути застосовані в біоінформатиці, як база для розробки нових моделей РНК та ДНК.
