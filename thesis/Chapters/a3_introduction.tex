%!TEX root = ../thesis.tex
% створюємо вступ
\textbf{Актуальність дослідження.} Бурхливий розвиток штучного інтелекту сьогодні
стимулює дослідження структури таких об'єктів, як природні мови, ланцюжки ДНК та інші структури, що
мають певну організованість на вищому рівні. Також дослідження властивостей та побудова ефективних
моделей мови є важливим аспектом криптоаналізу, оскільки це може дати криптоаналітику додаткову
інформацію, щодо структури відкритого тексту, до якого застосовується певний криптографічні перетворення.

\textbf{Метою дослідження} є побудова нової моделі природних мов, що дасть змогу краще
використовувати структуру мов для методів криптоаналізу. Для досягнення мети необхідно
вирішити наступні \textbf{завдання дослідження}:


\begin{enumerate}
\item провести огляд опублікованих робіт за тематикою дослідження;
\item розробити модель стохастичних регулярних граматик на основі прихованих моделей Маркова;
\item побудувати модель стохастичних контекстно-вільних граматик на основі узагальнення прихованих моделей Маркова та дослідити властивості цієї моделі;
\item розробити програмну реалізацію моделей;
\item перевірити узгодженість моделей емпіричним шляхом.
\end{enumerate}

\emph{Об'єктом дослідження} є формальна модель природних мов.

\emph{Предметом дослідження} є побудова поліпшеної моделі природної мови на основі стохастичної контекстно-вільної граматики в нормальній формі Грейбах.

При розв’язанні поставлених завдань використовувались такі \emph{методи дослідження}: методи теорії формальних мов та граматик,
теорії складності, теорії автоматів, теорії імовірності, математичної лінгвістики та машинного навчання.

\textbf{Наукова новизна} отриманих результатів полягає в вперше запропонованій формальній моделі мови на основі узагальнення прихованої моделі Маркова; в доведенні еквівалентності стохастичних контекстно-вільних граматик та прихованих моделей Маркова другого порядку за спостереженнями зі стеком.

\textbf{Практичне значення} результатів полягає у використанні поліпшеної моделі природних мов в методах криптоаналізу. Запропонована модель має меншу складність практичної реалізації у порівнянні з наявними сучасними моделями та краще алгоритмізується.

\textbf{Апробація результатів та публікації.} Результати були апробовані на конференціях:\\
\begin{enumerate}
  \item Yevhen Hrubiian. New Probabilistic Model of Stochastic Context-Free Grammars in Greibach Normal Form // Інформаційні технології
та комп’ютерне моделювання, Івано-Франківськ - 2017
  \item Євген Грубіян. Еквівалентність стохастичних контекстно-вільних граматик та прихованих моделей Маркова зі стеком // Безпека інформації у інформаційно-телекомунікаційних системах, Буча - 2017
\end{enumerate}
