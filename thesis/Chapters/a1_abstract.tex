%!TEX root = ../abstract.tex
%!TeX spellcheck = en-US,uk-UA,ru-RU
\abstractUkr
%\chapter*{Реферат}
% далі пишемо текст анотації

% анотація повинна починатися інформацією про структуру роботи
Роботу виконано на 70 аркушах, вона містить 1 додаток та перелік посилань на використані джерела з 22 найменувань.

% далі потрібно вказати мету роботи
Метою дипломної роботи є побудова нової моделі природних мов, що дасть змогу якісно
використовувати структуру мов для методів криптоаналізу.

\emph{Об'єктом дослідження} є формальна модель природних мов.

\emph{Предметом дослідження} є побудова нової моделі природної мови на основі стохастичної контекстно-вільні граматики в нормальній формі Грейбах.

В роботі зроблено огляд класичних і сучасних підходів до моделювання мови, проведено паралелі з іншими галузями науки, де використовується теорія мов та граматик. Запропоновано нову модель відкритого тексту(природної мови) на основі прихованої моделі Маркова другого порядку за спостереженнями зі стеком. Досліджено її властивості і доведена еквівалентність розробленої моделі та стохастичних контекстно-вільних граматик. Запропоновано поліноміальний алгоритм зведення стохастичної контекстно-вільної граматики до введеної моделі. Розроблено програмний пакет з реалізацією запропонованої моделі та емпіричним шляхом підтверджено узгодженість побудованої моделі та стохастичних контекстно-вільних граматик.

Запропонована модель вимагає розробки нових алгоритмів навчання стохастичних граматик.

% наприкінці анотації потрібно зазначити ключові слова
\MakeUppercase{стохастична контекстно-вільна граматика, прихована модель Маркова, модель відкритого тексту, природні мови}

%\chapter*{РЕФЕРАТ}
\abstractRus

Дипломная работа выполнена на 70 листах, она содержит 1 приложение и список ссылок на использованные источники с 22 наименований.

Целью дипломной работы является построение новой модели естественных языков, что дает возможность качественно использовать структуру языков для методов криптоанализа.

\textit{Объектом исследования} является формальная модель естественных языков.

\textit{Предметом исследования} является построение новой модели естественного языка на основе стохастических контекстно-свободных грамматик в нормальной форме Грейбах.

В работе проведен обзор классических и современных подходов к моделированию языка, проведены параллели с другими областями науки, где используется теория языков и грамматик. Предложено новую модель открытого текста(естественного языка) основанную на скрытой модели Маркова второго порядка по наблюдениям со стеком. Исследовано ее свойства и доказана эквивалентность разработанной модели и стохастических контекстно-свободных грамматик. Предложено полиномиальный алгоритм сведения стохастической контекстно-свободной грамматики к введенной модели. Разработан программный пакет с реализацией предложенной модели и эмпирическим путем подтверждена согласованность построенной модели c стохастическими контекстно-свободными грамматиками.

Предложенная модель требует разработки новых алгоритмов обучения стохастических грамматик.

\MakeUppercase{стохастическая контекстно-свободная грамматика, скрытая модель Маркова, модель открытого текста, естественные языки}

% створюємо анотацію англійською мовою
%\chapter*{Abstract}
\abstractEng

% анотація повинна починатися інформацією про структуру роботи
% (кількість аркушів (БЕЗ ДОДАТКІВ!), додатків, посилань, рисунків і таблиць)
The thesis is presented in 70 pages. It contains 1 appendix and bibliography of 22 references.

% далі потрібно вказати мету роботи
The target of the thesis is construction of an effective model of natural languages which allows using a structure of language for cryptanalysis more efficiently.

\textit{The object} is a formal model of natural languages.

\textit{The subject} is developing a new model of a natural language based on stochastic context-free grammars in Greibach normal form.

We made an overview of classical and modern approaches for language modeling and drew parallels to other spheres of science where a theory of languages and grammars is used. We developed the probabilistic model of stochastic context-free language based on hidden Markov model of the second order with stack and proved equivalence between them. Also, we proposed the polynomial algorithm for transformation of a stochastic context-free grammar to the proposed model. We developed the software which implements proposed model and empirically confirmed concordance between proposed model and stochastic context-free grammars.

Proposed model requires developing of new learning algorithms.
% далі потрібно вказати розглянуті методи та критерії, за яким вибрано один із них


% далі потрібно коротко викласти суть роботи


% далі потрібно подати відомості про апробацію роботи


% наприкінці анотації потрібно зазначити ключові слова
\MakeUppercase{stochastic context-free grammar, hidden Markov model, plaintext model, natural languages}


% Не прибирайте даний рядок
\clearpage
